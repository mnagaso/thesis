% This is the abstract of my dissertation, in English

\begin{center}
\underline{\textbf{\Large SUMMARY}}
\end{center}

The ASTRID project, a French sodium-cooled nuclear reactor of 4th generation, is currently under development by the French Alternative Energies and Atomic Energy
Center (CEA). In this project, development of monitoring techniques for a nuclear reactor in operation is identified as an important issue to improve
the plant safety. The use of ultrasonic measurement techniques (e.g. thermometry, visualization of internal objects) is regarded as a powerful inspection tool
for sodium-cooled fast reactors, including ASTRID due to the opacity of liquid sodium.

Inside a sodium cooling circuit, heterogeneity of the medium occurs because of a complex flow state, especially when the reactor is in operation, and then the
effects of this heterogeneity on acoustic wave propagation are not negligible. Thus, it is necessary to carry out verification experiments for development of
component technologies, and such kind of experiments using liquid sodium may be relatively large-scale, i.e., difficult and expensive. This is a reason why numerical simulation methods
are essential before performing real laboratory experiments, or in addition to the number of experimental results, which is necessarily limited due to their
difficulty and cost. Though various numerical methods have been
applied to model wave propagation in liquid sodium, the community still does not have a verified and fully tested full-wave method for
numerical modeling of wave propagation in large-scale three-dimensional heterogeneous sodium reactors.
Moreover, inside of a reactor core i.e. in a complex acoustic-elastic coupled region, it is also difficult to simulate such problems with conventional ray-based
methods.

The objective of the study in the thesis is to contribute to solving these two points by resorting to a three-dimensional spectral-element method, which is a high-order
time-domain finite-element method that we will show to be suitable to model our targets (i.e. sodium coolant inside a nuclear reactor) more accurately
than more classical numerical simulation methods.

We will first study the development potential of ultrasonic thermometry in a liquid fluctuating sodium environment similar to that present in a Sodium-cooled
Fast Reactor, and thus investigate if and how ultrasonic thermometry could be used to monitor the sodium flow at the outlet of the reactor core. We will obtain
clear time-of-flight variations in the case of a small temperature difference of one percent in the case of a static temperature gradient as well as in the
presence of a random fluctuation of the temperature field in the turbulent flow. We will verify that small temperature variations in the sodium flow of e.g.
about \SI{1}{\percent} of the sodium temperature, i.e. about 5 degrees Celsius, can have a reliably-measurable acoustic signature. For this calculation, the
target domain will be modeled as a two-dimensional medium, and a Gaussian random process will be applied to generate fluctuations of temperature in the liquid sodium.

To investigate 3D heterogeneity and more realistic temperature fields in the medium, in a second part of the thesis we will carry out a numerical study
for 3D models of the reactor core. To represent the
heterogeneity of liquid sodium, a four-dimensional temperature field (three spatial and one temporal dimension) calculated by computational fluid dynamics based on
a Large-Eddy Simulation performed by CEA STMF will be applied instead of using a cheaper, more classical method such as e.g. a Gaussian random
process. We will show that based on that numerical experiment we will be able to analyze the 3D effects of realistic
heterogeneity in the propagation medium on the propagation of acoustic waves in liquid sodium, in a jet-mixing experiment called PLAJEST.
We will show that from acoustic measurements we can deduce information relevant to design studies in thermal-hydraulics: frequency of temperature variations, delimitation of the zone of greater fluctuation of temperature,
and average temperature with respect to altitude.
