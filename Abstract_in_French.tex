% This is the abstract of my dissertation, in French

\begin{center}
\underline{\textbf{\Large R\'ESUM\'E}}
\end{center}

Le projet ASTRID, réacteur nucléaire français de quatrième génération refroidi au sodium, est actuellement en cours de développement par
le Commissariat à l'énergie atomique et aux énergies alternatives (CEA). Dans ce projet, le développement de techniques de surveillance pour un réacteur nucléaire en
fonctionnement est identifié comme un problème majeur pour augmenter la sécurité du réacteur. L'utilisation de techniques de mesure par
ultrasons (par exemple thermométrie, visualisation d'objets internes) est considérée comme un puissant outil d'inspection des réacteurs rapides refroidis
au sodium, y compris ASTRID en raison de l'opacité du sodium liquide.

\`A l'intérieur d'un circuit de refroidissement au sodium, l'hétérogénéité du milieu se produit du fait de l'état d'écoulement complexe, notamment
lorsque le réacteur est en fonctionnement, et les effets de cette hétérogénéité sur la propagation des ondes acoustiques ne sont pas négligeables.
Ainsi, il est nécessaire d'effectuer des expériences de vérification pour les développements de technologies pour les composants, sachant que de telles
expériences utilisant du sodium liquide peuvent être des expériences à relativement grande échelle. C'est pourquoi les méthodes de simulation numérique
sont essentielles avant d'effectuer des expériences physiques en laboratoire, et en complément des résultats expérimentaux, qui sont nécessairement limités en nombre.
Bien que diverses méthodes numériques
aient été utilisées pour modéliser la propagation d'ondes acoustiques dans le sodium liquide, la communauté n'a toujours pas de méthode de modélisation de formes d'ondes complètes
qui soient bien validées dans le cas de modèles tridimensionnels de grande taille présentant des hétérogénéités.
De plus, à l'intérieur d'un coeur de réacteur, c'est-à-dire une région couplée acoustique-élastique
complexe, il est également difficile de simuler de manière précise de tels problèmes avec des techniques numériques qui soient basées sur du tracé de rais conventionnel.

L'objectif de l'étude menée dans le cadre de ma thèse est donc de contribuer à résoudre ces deux points en appliquant une méthode de calcul tridimensionnelle
par la technique numérique des éléments spectraux, qui est une
méthode d'éléments finis d'ordre élevé calculée dans le domaine temporel, qui peut modéliser nos objets d'étude (par exemple le milieu caloporteur sodium à l'intérieur du réacteur
nucléaire) de manière plus précise que les méthodes de simulation plus classiques.

Nous étudierons d'abord le potentiel de développement de la thermométrie ultrasonique dans un environnement sodium liquide fluctuant similaire à celui d'un
réacteur rapide refroidi au sodium, et étudierons si et comment la thermométrie ultrasonique peut être utilisée pour surveiller le flux de sodium à
la sortie du coeur du réacteur. Nous obtiendrons des variations de temps de vol claires dans le cas d'une faible différence de température de 1\% dans le cas
d'un gradient de température statique ainsi qu'en présence d'une fluctuation aléatoire du champ de température dans le flux turbulent. Nous
vérifierons que de petites variations de température dans le flux de sodium de par exemple environ 1\% de la température du sodium, c'est-à-dire environ 5 degrés
Celsius, peuvent avoir une signature acoustique mesurable de manière fiable. Pour ce calcul, le domaine cible sera modélisé comme un processus aléatoire
bidimensionnel et Gaussien appliqué pour générer une fluctuation de la température dans le sodium liquide.

Afin d'étudier l'hétérogénéité tridimensionnelle et des champs de température plus réalistes dans le milieu, dans un deuxième temps dans cette thèse nous effectuerons une seconde étude
numérique, cette fois-ci à trois dimensions. Pour représenter l'hétérogénéité du sodium liquide, nous appliquerons un champ de température quadridimensionnel (trois dimensions spatiale et une dimension temporelle)
calculé par modélisation numérique en dynamique des fluides avec une simulation LES (Large-Eddy Simulation) réalisée par CEA STMF au lieu d'une méthode conventionnelle
plus classique et moins chère (par exemple un processus aléatoire Gaussien). Nous montrerons qu'à partir de cette expérience numérique tridimensionnelle, nous
serons en mesure d'analyser les effets tridimensionnels de l'hétérogénéité réaliste dans le milieu de propagation sur les ondes acoustiques se propageant dans le sodium liquide,
dans une expérience de mélange de jets appelée PLAJEST.
Nous montrerons également que l'on peut déduire des mesures acoustiques des informations pertinentes pour des études de conception dans le domaine de la thermo-hydraulique :
fréquence des fluctuations de température, délimitation de la zone de plus fortes fluctuations de température, et température moyenne en fonction de l'altitude.

